% Options for packages loaded elsewhere
\PassOptionsToPackage{unicode}{hyperref}
\PassOptionsToPackage{hyphens}{url}
%
\documentclass[
  11pt,
  ignorenonframetext,
]{beamer}
\usepackage{pgfpages}
\setbeamertemplate{caption}[numbered]
\setbeamertemplate{caption label separator}{: }
\setbeamercolor{caption name}{fg=normal text.fg}
\beamertemplatenavigationsymbolsempty
% Prevent slide breaks in the middle of a paragraph
\widowpenalties 1 10000
\raggedbottom
\setbeamertemplate{part page}{
  \centering
  \begin{beamercolorbox}[sep=16pt,center]{part title}
    \usebeamerfont{part title}\insertpart\par
  \end{beamercolorbox}
}
\setbeamertemplate{section page}{
  \centering
  \begin{beamercolorbox}[sep=12pt,center]{part title}
    \usebeamerfont{section title}\insertsection\par
  \end{beamercolorbox}
}
\setbeamertemplate{subsection page}{
  \centering
  \begin{beamercolorbox}[sep=8pt,center]{part title}
    \usebeamerfont{subsection title}\insertsubsection\par
  \end{beamercolorbox}
}
\AtBeginPart{
  \frame{\partpage}
}
\AtBeginSection{
  \ifbibliography
  \else
    \frame{\sectionpage}
  \fi
}
\AtBeginSubsection{
  \frame{\subsectionpage}
}

\usepackage{amsmath,amssymb}
\usepackage{iftex}
\ifPDFTeX
  \usepackage[T1]{fontenc}
  \usepackage[utf8]{inputenc}
  \usepackage{textcomp} % provide euro and other symbols
\else % if luatex or xetex
  \usepackage{unicode-math}
  \defaultfontfeatures{Scale=MatchLowercase}
  \defaultfontfeatures[\rmfamily]{Ligatures=TeX,Scale=1}
\fi
\usepackage{lmodern}
\usetheme[]{AnnArbor}
\usecolortheme{seahorse}
\ifPDFTeX\else  
    % xetex/luatex font selection
\fi
% Use upquote if available, for straight quotes in verbatim environments
\IfFileExists{upquote.sty}{\usepackage{upquote}}{}
\IfFileExists{microtype.sty}{% use microtype if available
  \usepackage[]{microtype}
  \UseMicrotypeSet[protrusion]{basicmath} % disable protrusion for tt fonts
}{}
\makeatletter
\@ifundefined{KOMAClassName}{% if non-KOMA class
  \IfFileExists{parskip.sty}{%
    \usepackage{parskip}
  }{% else
    \setlength{\parindent}{0pt}
    \setlength{\parskip}{6pt plus 2pt minus 1pt}}
}{% if KOMA class
  \KOMAoptions{parskip=half}}
\makeatother
\usepackage{xcolor}
\newif\ifbibliography
\setlength{\emergencystretch}{3em} % prevent overfull lines
\setcounter{secnumdepth}{5}


\providecommand{\tightlist}{%
  \setlength{\itemsep}{0pt}\setlength{\parskip}{0pt}}\usepackage{longtable,booktabs,array}
\usepackage{calc} % for calculating minipage widths
\usepackage{caption}
% Make caption package work with longtable
\makeatletter
\def\fnum@table{\tablename~\thetable}
\makeatother
\usepackage{graphicx}
\makeatletter
\def\maxwidth{\ifdim\Gin@nat@width>\linewidth\linewidth\else\Gin@nat@width\fi}
\def\maxheight{\ifdim\Gin@nat@height>\textheight\textheight\else\Gin@nat@height\fi}
\makeatother
% Scale images if necessary, so that they will not overflow the page
% margins by default, and it is still possible to overwrite the defaults
% using explicit options in \includegraphics[width, height, ...]{}
\setkeys{Gin}{width=\maxwidth,height=\maxheight,keepaspectratio}
% Set default figure placement to htbp
\makeatletter
\def\fps@figure{htbp}
\makeatother

\usepackage{booktabs}
\usepackage{longtable}
\usepackage{array}
\usepackage{multirow}
\usepackage{wrapfig}
\usepackage{float}
\usepackage{colortbl}
\usepackage{pdflscape}
\usepackage{tabu}
\usepackage{threeparttable}
\usepackage{threeparttablex}
\usepackage[normalem]{ulem}
\usepackage{makecell}
\usepackage{xcolor}
\usepackage{amsmath, amssymb, bbm, amstext, array, listings, mathtools, caption, color, graphics, ulem, caption, changepage, atbegshi, soul}
\newcommand\E{\mathbb{E}}
\newcommand\V{\mathbb{V}}
\hypersetup{colorlinks=true,linkcolor=red}
\usepackage{ulem}
\pdfstringdefDisableCommands{\let\sout\relax}
\makeatletter
\makeatother
\makeatletter
\makeatother
\makeatletter
\@ifpackageloaded{caption}{}{\usepackage{caption}}
\AtBeginDocument{%
\ifdefined\contentsname
  \renewcommand*\contentsname{Table of contents}
\else
  \newcommand\contentsname{Table of contents}
\fi
\ifdefined\listfigurename
  \renewcommand*\listfigurename{List of Figures}
\else
  \newcommand\listfigurename{List of Figures}
\fi
\ifdefined\listtablename
  \renewcommand*\listtablename{List of Tables}
\else
  \newcommand\listtablename{List of Tables}
\fi
\ifdefined\figurename
  \renewcommand*\figurename{Figure}
\else
  \newcommand\figurename{Figure}
\fi
\ifdefined\tablename
  \renewcommand*\tablename{Table}
\else
  \newcommand\tablename{Table}
\fi
}
\@ifpackageloaded{float}{}{\usepackage{float}}
\floatstyle{ruled}
\@ifundefined{c@chapter}{\newfloat{codelisting}{h}{lop}}{\newfloat{codelisting}{h}{lop}[chapter]}
\floatname{codelisting}{Listing}
\newcommand*\listoflistings{\listof{codelisting}{List of Listings}}
\makeatother
\makeatletter
\@ifpackageloaded{caption}{}{\usepackage{caption}}
\@ifpackageloaded{subcaption}{}{\usepackage{subcaption}}
\makeatother
\makeatletter
\@ifpackageloaded{tcolorbox}{}{\usepackage[skins,breakable]{tcolorbox}}
\makeatother
\makeatletter
\@ifundefined{shadecolor}{\definecolor{shadecolor}{rgb}{.97, .97, .97}}
\makeatother
\makeatletter
\makeatother
\makeatletter
\makeatother
\ifLuaTeX
  \usepackage{selnolig}  % disable illegal ligatures
\fi
\IfFileExists{bookmark.sty}{\usepackage{bookmark}}{\usepackage{hyperref}}
\IfFileExists{xurl.sty}{\usepackage{xurl}}{} % add URL line breaks if available
\urlstyle{same} % disable monospaced font for URLs
\hypersetup{
  pdftitle={Lectures on causal inference and experimental methods},
  pdfauthor={Macartan Humphreys},
  hidelinks,
  pdfcreator={LaTeX via pandoc}}

\title{Lectures on causal inference and experimental methods}
\author{Macartan Humphreys}
\date{}

\begin{document}
\frame{\titlepage}
\ifdefined\Shaded\renewenvironment{Shaded}{\begin{tcolorbox}[boxrule=0pt, breakable, enhanced, frame hidden, sharp corners, interior hidden, borderline west={3pt}{0pt}{shadecolor}]}{\end{tcolorbox}}\fi

\hypertarget{secoutline}{%
\subsection{Getting started}\label{secoutline}}

\begin{frame}[fragile]{Getting started}
\begin{itemize}
\tightlist
\item
  General aims and structure
\item
  Expectations
\item
  Pointers for exercises
\item
  Quick \texttt{DeclareDesign} intro
\end{itemize}
\end{frame}

\begin{frame}{Aims}
\protect\hypertarget{aims}{}
\begin{itemize}
\tightlist
\item
  Deep understanding of key ideas in causal inference
\item
  Transportable tools for understanding how to evaluate and improve
  design
\item
  Applied skills for design and analysis
\item
  Exposure to open science practices
\item
  Deeper dive into some specific topics (see survey)
\end{itemize}
\end{frame}

\begin{frame}{Syllabus and resources}
\protect\hypertarget{syllabus-and-resources}{}
\begin{itemize}
\tightlist
\item
  \url{https://macartan.github.io/ci} for all resources
\item
  \url{https://macartan.github.io/ci/syllabus.pdf}
\item
  Git repo \url{https://github.com/macartan/ci}
\item
  \href{https://cloud.wzb.eu/apps/forms/s/8QmokT5GeQfkmkrBHDzrzN4j}{Student
  survey}: Please fill this out today.
\end{itemize}
\end{frame}

\begin{frame}[fragile]{The topics}
\protect\hypertarget{the-topics}{}
Day 1: Intro

\begin{itemize}
\tightlist
\item
  \protect\hyperlink{secoutline}{1.1 Course outline, tools}
\item
  \protect\hyperlink{secdd}{1.2 Introduction to \texttt{DeclareDesign}}
\end{itemize}

Day 2: Causality

\begin{itemize}
\tightlist
\item
  \protect\hyperlink{seccausality}{2.1 Fundamental problems and
  solutions}
\item
  \protect\hyperlink{secestimands}{2.2 Inquiries and identification}
\end{itemize}
\end{frame}

\begin{frame}{The topics}
\protect\hypertarget{the-topics-1}{}
Day 3: Estimation and Inference

\begin{itemize}
\tightlist
\item
  \protect\hyperlink{secfisher}{3.1 Frequentist}
\item
  \protect\hyperlink{secbayes}{3.2 Bayesian}
\end{itemize}

Day 4:

\begin{itemize}
\tightlist
\item
  \protect\hyperlink{secdesign}{4.1 Experimental Design
  (esp.~assignment)}
\item
  \protect\hyperlink{secdiagnosis}{4.2 Design evaluation (esp.~power)}
\end{itemize}

Day 5:

\begin{itemize}
\tightlist
\item
  \protect\hyperlink{citopics}{5.1 Topics and techniques}
\item
  \protect\hyperlink{openscience}{5.2 Open science}
\end{itemize}
\end{frame}

\begin{frame}{Expectations}
\protect\hypertarget{expectations}{}
\begin{itemize}
\tightlist
\item
  5 tasks
\item
  (Required) Work in four ``exercise teams'': 1 team (and typically 2
  exercises) per session \(\times 4\)
\item
  (Optional) Prepare a research design or short paper, perhaps building
  on existing work. Typically this contains:

  \begin{itemize}
  \tightlist
  \item
    a problem statement
  \item
    a description of a method to address the problem
  \item
    analytic or simulation based results describing properties of the
    solution
  \item
    a discussion of implications for practice.
  \end{itemize}
\end{itemize}

A passing paper will illustrate subtle features of a method; a good
paper will identify unknown properties of a method; en excellent paper
will develop a new method.

\begin{itemize}
\tightlist
\item
  Plus general reading and participation.
\end{itemize}
\end{frame}

\begin{frame}[fragile]{Exercise team job}
\protect\hypertarget{exercise-team-job}{}
Teams should prepare 15 - 20 minute presentations on set puzzles.
Typically the task is to:

\begin{itemize}
\item
  Take a puzzle
\item
  Declare and diagnose a design that shows the issue under study
  (e.g.~some estimator produces unbiased estimates under some condition)
\item
  Modify the design to show behavior when conditions are violated
\item
  Share a report with the class. Best in self-contained documents for
  easy third party viewing. e.g.~\texttt{.html} via \texttt{.qmd} or
  \texttt{.Rmd}
\item
  Presentations should be about 10 minutes for a given puzzle.
\end{itemize}
\end{frame}

\begin{frame}{Good coding rules}
\protect\hypertarget{good-coding-rules}{}
\begin{itemize}
\tightlist
\item
  \url{https://bookdown.org/content/d1e53ac9-28ce-472f-bc2c-f499f18264a3/code.html}
\item
  \url{https://www.r-bloggers.com/2018/09/r-code-best-practices/}
\end{itemize}
\end{frame}

\begin{frame}[fragile]{Good coding rules}
\protect\hypertarget{good-coding-rules-1}{}
\begin{itemize}
\tightlist
\item
  Metadata first
\item
  Call packages at the beginning: use \texttt{pacman}
\item
  Put options at the top
\item
  Call all data files once, at the top. Best to call directly from a
  public archive, when possible.\\
\item
  Use functions and define them at the top: comment them; useful
  sometimes to illustrate what they do
\item
  Replicate first, re-analyze second. Use sections.
\item
  (For replications) Have subsections named after specific tables,
  figures or analyses
\end{itemize}
\end{frame}

\begin{frame}[fragile]{Aim}
\protect\hypertarget{aim}{}
\begin{itemize}
\item
  First best: If someone has access to your \texttt{.Rmd}/\texttt{.qmd}
  file they can hit render or compile and the whole thing reproduces
  first time. So: Nothing local, everything relative: so please do not
  include hardcoded paths to your computer
\item
  But: often you need ancillary files for data and code. That's OK but
  aims should still be that with a self contained folder someone can
  open a \texttt{main.Rmd} file, hit compile and get everything. I
  usually have an \texttt{input} and an \texttt{output} subfolder.
\end{itemize}
\end{frame}

\begin{frame}[fragile]{Collaborative coding / writing}
\protect\hypertarget{collaborative-coding-writing}{}
\begin{itemize}
\tightlist
\item
  Do not get in the business of passing attachments around
\item
  Documents in some cloud: \texttt{git}, osf, Dropbox, Drive, Nextcloud
\item
  General rule: only post non sensitive, non proprietary material
\item
  Share self contained folders; folders contain a small set of live
  documents plus an archive. Old versions of documents are in archive.
  Only one version of the most recent document is in a main folder.
\item
  Data is self contained folder (\texttt{in}) and is never edited
  directly
\item
  Update to github frequently
\end{itemize}
\end{frame}



\end{document}
